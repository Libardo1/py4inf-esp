% The contents of this file is 
% Copyright (c) 2009- Charles R. Severance, All Righs Reserved

\chapter{Prefacio}

\section*{Python para Inform\'atica: La remezcla de un libro abierto}

Es natural que para aquellos acad\'emicos que continuamente han escuchado ``publique o perezca'' quieran siempre crear algo desde el comienzo. Este libro es  un 
experimento al no iniciarse de cero sino m\'as bien ``remezclando''
el libro titulado
\emph{Pensar Python: C\'omo pensar como un cient\'ifico de c\'omputo}
escrito por Allen B. Downey, Jeff Elkner y otros.

En diciembre de 2009 me estaba preparando para ense\~nar
{\bf SI502 - Programaci\'on en red (\textit{Networked Programming})} en la Universidad de Michigan,
por quinta vez seguida, y decid\'i que era hora de escribir una gu\'ia did\'actica para Python que se enfocara en la exploraci\'on de datos en vez de en entender algoritmos y abstracciones.
Mi meta en SI502 es ense\~narle a la gente destrezas para manejar datos en la vida diaria utilizando Python. Pocos de mis estudiantes planeaban ser programadores de computadoras profesionales. La mayor\'ia, en cambio, planeaban ser bibliotecarios, administradores, abogados, bi\'ologos, economistas, etc. y simplemente quer\'ian desarrollar destrezas en el manejo de la tecnolog\'ia en su campo.

Parec\'ia como si no se pudiera encontrar el libro perfecto orientado al manejo de los datos en Python para el curso que planeaba ense\~nar, as\'i que decid\'i yo mismo escribir tal libro. 
Afortunadamente, en una reuni\'on de la facultad tres semanas antes de empezar a escribir mi nuevo libro desde el comienzo, durante una jornada de vacaciones, el Dr. Atul Prakash me mostr\'o el libro \emph{Pensar Python} el cual \'el hab\'ia usado para ense\~nar su curso de Python durante ese semestre.  
Es un texto de Inform\'atica muy bien escrito con un \'enfasis en peque\~nas, directas explicaciones y facilidad de aprendizaje.  

La estructura global se ha cambiado de modo que se logre hacer el an\'alisis de los datos tan r\'apido como sea posible y, desde el principio, tiene una serie de ejemplos y ejercicios sobre el an\'alisis de los datos.

Los cap\'itulos 2-10 son similar al libro \emph{Pensar Python}
pero han sufrido grandes cambios. Los ejemplos numerados y los ejercicios se han reemplazado con ejercicios que tienen que ver con el manejo de datos. Los temas se presentan en el orden necesario para ir construyendo cada vez m\'as soluciones sofisticadas de an\'alisis de datos. Algunos temas como {\tt try} y
{\tt except} se han desplazado para presentarlos como parte del cap\'itulo sobre afirmaciones condicionales (\textit{conditionals}). A las funciones (\textit{Functions}) se les da un tratamiento ligero hasta que sean necesarias para manejar la complejidad de un programa en vez de introducirlas como concepto abstracto temprano en la lecci\'on. Casi todas las funciones definidas por el usuario (\textit{user-defined functions}) han sido removidas de los ejemplos y ejercicios fuera del Cap\'itulo 4. La palabra ``recursi\'on''\footnote{A excepci\'on de esta l\'inea.} no se menciona en el libro.

El material de los cap\'itulos 1 y 11-16 es completamente nuevo, con un enfoque al uso en el mundo real y se presentan ejemplos sencillos de Python para el an\'alisis de datos, incluyendo expresiones regulares (\textit{regular expressions}) en la realizaci\'on de b\'usquedas y an\alisis sint\'actico en la automatizaci\'on de tareas en su computador, recuperaci\'on de datos a trav\'es del network, recabaci\'on (\textit{scraping}) de datos de p\'aginas web, utilizando servicios web, en el an\alisis sint\'actico (\textit{parsing} de datos en XML y JSON, y en la creaci\'on y uso de bases de datos estructaradas en leguage de consulta (\textit{Structured Query Language}).

El objetivo primordial de estos cambios es pasar de un enfoque de Ciencias de computaci\'on a Inform\'atica y solo incluir temas de tecnolog\'ia \'utiles aun si alguien no escoge ser programador como profesi\'on.

Los estudiantes que encuentren este libro interesante y quieran continuar explorando deben mirar el libro de Allen B. Downey's \emph{Pensar Python}. Siendo que hay cruce en muchos de los temas de los dos libros,
los estudiantes rapidamente adquirir\'an destrezas en \'areas adicionales de programaci\'on t\'ecnica y pensamiento algor\'itmico cubierto en \emph{Pensar Python}, y siendo que los dos libros se han escrito en un estilo similar, usted podr\'a avanzar r\'apidamente a trav\'es del libro \emph{Pensar Python} con un m\'inimo esfuerzo.

\index{Creative Commons License}
\index{CC-BY-SA}
\index{BY-SA}
Como titular de los derechos de autor de \emph{Pensar Python},
Allen me ha dado permiso de cambiar la licencia del libro en lo que al material de su obra que aparece en este libro se refiere, para convertirla de GNU Free Documentation License 
a la m\'as reciente licencia de Creative Commons Reconocimiento --- Compartir. Esto se adecua al cambio general en la documentaci\'on abierta de licencias de GFDL a CC-BY-SA (i.e. Wikipedia).
Al usar la licencia CC-BY-SA se mantiene la fuerte tradici\'on de \textit{copyleft} mientras  que se se abre aun m\'as las oportunidades para nuevos autores de usar este material como lo consideren apropiado.

Pienso que este libro es un ejemplo de porqu\'e el material de dominio p\'ublico es tan importante para el futuro de la educaci\'on y quiero agradecerle a Allen B. Downey y a la imprenta de la Universidad de Cambridge por su decisi\'on visionaria de facilitar este libro bajo la licencia de derechos de autor de dominio p\'ublico, \textit{open Copyright}. Espero que est\'en contentos con los resultados de mi esfuerzo y espero que usted, el lector, quede satisfecho con \emph{nuestro} esfuerzo colectivo.

Me gustar\'ia agradecerle a Allen B. Downey y a Lauren Cowles por su ayuda,
paciencia y gu\'ia en el manejo y resoluci\'on de los asuntos realacionados al \textit{copyright} de este libro.

Charles Severance\\
www.dr-chuck.com\\
Ann Arbor, MI, USA\\
Septiembre 9, 2013

Charles Severance es profesor cl\'inico asociado de la facultad de Inform\'atica de la Universidad de Michigan.

\clearemptydoublepage

% TABLE OF CONTENTS
\begin{latexonly}

\tableofcontents

\clearemptydoublepage

\end{latexonly}

% START THE BOOK
\mainmatter

